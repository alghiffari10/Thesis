\newpage
\chapter{Pendahuluan} \label{Bab I}

\section{Latar Belakang} \label{I.Latar Belakang}
Kartu kredit merupakan alat pembayaran pengganti uang tunai dalam melakukan transaksi. Kemudahan untuk tidak perlu membawa uang tunai meningkatkan penggunaan kartu kredit di seluruh dunia \cite{suryohadibroto1987surat}. Namun, di balik balik segala kemudahan dan keuntungan yang manjadikan ketergantungan terhadap penggunaan kartu kredit, penggunaan kartu kredit tidak lepas dari risiko kejahatan, seperti penipuan transaksi \cite{hendarsyah2020analisis}. Berdasarkan data dari Experian 60\% dari pengguna kartu kredit pernah mengalami penipuan kartu kredit yang mana memberikan kerugian yang besar bagi pengguna kartu kredit dan juga agensi kartu kredit \cite{experian_2024} dan bahkan menurut analisis \textit{Merchant Cost Consulting} kerugian dari transaksi penipuan kartu kredit akan terus bertambah dan bisa mencapai 43 miliar dolar pada tahun 2026\cite{merchantcostconsulting_2024}. Tentu masalah ini akan menjadi masalah yang besar dimasa mendatang jika tidak ditangani dengan baik.

Penipuan kartu kredit bisa berupa penipuan kartu kredit internal dan external, penipuan kartu kredit internal bisa terjadi karena persetujuan antara pemegang kartu dan bank dengan menggunakan identitas palsu untuk melakukan penipuan sedangkan penipuan kartu kredit external melibatkan penggunaan kartu kredit curian untuk mendapatkan uang tunai\cite{chaudhary2012review}. Banyak sekali penelitian yang berfokus pada penipuan kartu kredit external dikarenakan menyumbang sebagian besar penipuan kartu kredit\cite{chase_jpmorgan_fraud_management}. Mendeteksi penipuan kartu kredit secara manual memakan waktu yang banyak dan tidak efisien, disebabkan \textit{big data} yang sudah tidak memungkinankan metode manual bisa dilakukan. yang mana, lembaga keuangan sekarang lebih fokus kepada \textit{computational methodologies} sebagai metode yang lebih efisien dalam mengatasi hal tersebut\cite{west2016intelligent}.\\
Data mining merupakan salah satu \textit{computational methodologies} terbaik untuk mendeteksi penipuan kartu kredit dengan cara membagi menjadi dua kelas yaitu transaksi normal dan transaksi penipuan\cite{ngai2011application}. Sistem pendeteksi penipuan kartu kredit berkerja berdasarkan pada analisis perilaku pengguna kartu kredit dalam melakukan transaksi. Banyak teknik atau algoritma yang digunakan untuk mendeteksi penipuan kartu kredit seperti menggunakan neural network\cite{georgieva2019using}, genetic algorithm\cite{benchaji2019using}, frequent itemset mining\cite{seeja2014fraudminer}, support vector machine\cite{kumar2022credit}, decision tree\cite{gaikwad2014credit} dan lain-lain. \\
Hasil dari teknik atau algoritma yang telah digunakan memberikan hasil yang berbeda-beda. untuk itu, ada beberapa cara metode dan tahap dalam memilih algoritma yang digunakan dalam memilih dalam pengembangan model \textit{machine learning} seperti identifikasi masalah dan melakukan komparasi algoritma yang berbeda-beda untuk kasus yang sama\cite{mniai2023novel}.\\
Masalah utama dalam membuat model pendeteksi penipuan kartu kredit ialah \textit{unbalance data} yang sangat berat antara transaksi penipuan dan transaksi normal yang membuat performa model menjadi sangat buruk dan perlu adanya preprocessing data terlebih dahulu seperti melakukan oversampling agar data yang diinputkan ke machine learningk menjadi \textit{balance} untuk menghasilkan performa terbaik\cite{mniai2023novel}.\\
Hasil riset-riset sebelumnya menunjukan bahwa algoritma \textit{esemble learning} merupakan algoritma terbaik dalam membuat model pendeteksi penipuan kartu kredit\cite{ningsih2022analisis}. \textit{Esemble learning} merupakan algoritma yang meningkatkan performa prediksi dan dapat mengatasi \textit{complex relationships} di sebuah machine learning dengan memanfaatkan \textit{multiple decision trees} salah satu contoh  algoritma esemble learning ialah random forest dan xgboost\cite{dietterich2002ensemble}.\\
Beberapa riset sebelumnya berfokus pada metode komparasi model machine learning guna menentukan model machine learning terbaik dalam membuat model pendeteksi penipuan kartu kredit\cite{mniai2023novel}. Namun, ada hal krusial yang jarang dijadikan fokus dalam penelitian yaitu process metode oversampling untuk mengatasi umbalance data, penggunaan metode oversampling yang berbeda-beda akan menghasilkan hasil performa yang berbeda-beda\cite{liu2004effect}. Metode oversampling yang digunakan dalam penelitian ini ialah \textit{synthetic minority over-sampling technique}(SMO
TE)\cite{chawla2002smote} dan \textit{adaptive synthetic sampling}(ADASYN)\cite{4633969} dikarenakan metode tersebut memberikan hasil yang selalu berbeda dan tidak saling mengungguli satu sama lain dan menghasilkan hasil yang baik dikasus yang berbeda-beda sesuai dengan pola dataset dan tujuan pembuatan model\cite{brandt2021comparative}.\\
Fokus utama dari riset penulis ialah melakukan \textit{comparative analysis} metode oversampling \textit{synthetic minority over-sampling}(SMOTE) dan \textit{adaptive synthetic sampling}(ADASYN) dengan menggunakan algoritma esemble learning seperti random forest dan xgboost dalam membuat model pendeteksi penipuan kartu kredit dengan perbandigan tes \textit{precision}, \textit{recall}, \textit{f1-score}, dan \textit{Matthews’s correlation coefficient} (MCC) \textit{metrics}. Dataset yang digunakan ialah dataset \textit{credit card fraud detection} yang dibuat dan dikumpulkan oleh grup machine learning dari \textit{Université Libre de Bruxelles}(ULB)\cite{dal2015calibrating}. Dataset ini berisi berisi 284807 transaksi kartu kredit yang dilakukan oleh pemegang kartu di  eropa  selama  dua  hari  dengan bobot  transaksi  normal  sebanyak  99,83\%  dan  fraud  sebanyak  0,17\% yang artinya dataset yang digunakan \textit{highly unbalanced}\cite{WinNT}. 
Riset kali ini akan memperluas penanganan \textit{unbalance data} penipuan kartu kredit untuk menentukan metode oversampling terbaik antara SMOTE dan ADASYN.

\section{Rumusan masalah} \label{I.Rumusan Masalah}
Berdasarkan latar belakang yang ada, rumusan masalah dalam penelitian ini sebagai berikut:

\begin{enumerate}[noitemsep]
        \item Bagaimana cara membangun model pendeteksi penipuan kartu kredit dengan menggunakan algoritma Random Forest dan XGBoost dengan metode oversampling SMOTE dan ADASYN?
        \item Bagaimana kinerja metode oversampling SMOTE dan ADASYN dalam meningkatkan akurasi model Random Forest dan XGBoost dalam mendeteksi penipuan kartu kredit?
\end{enumerate}

\section{Tujuan} \label{I.Tujuan}
Adapun tujuan dari penelitian ini adalah sebagai berikut:
\begin{enumerate}[noitemsep]
        \item Merancang dan membangun model pendeteksi penipuan kartu kredit dengan menggunakan algoritma Random Forest dan XGBoost dengan metode oversampling SMOTE dan ADASYN
        \item Mengetahui kinerja model dengan oversampling SMOTE dan ADASYN dalam meningkatkan akurasi model Random Forest dan XGBoost dalam mendeteksi penipuan kartu kredit
\end{enumerate}

\section{Batasan Masalah} \label{I.Batasan}
Batasan-batasan permasalahan dalam penelitian ini mencakup hal-hal berikut: 
\begin{enumerate}[noitemsep]
    \item Penelitian ini hanya menggunakan  dataset \textit{Credit Card Fraud Detection} dari kaggle dengan pengumpulannya berdasarkan hasil dari \textit{research collaboration} Université Libre de Bruxelles.
    \item Penelitian ini hanya akan menganalisis dua metode oversampling, yaitu SMOTE dan ADASYN, tanpa membandingkannya dengan metode oversampling atau undersampling lainnya.
    \item Analisis ini akan fokus pada dua algoritma machine learning, yaitu Random Forest dan Advanced Gradient Boosting Techniques seperti XGBoost tanpa mempertimbangkan algoritma lain seperti Support Vector Machine, Neural Networks, atau k-Nearest Neighbors.
\end{enumerate}

\section{Manfaat Penelitian} \label{I.Manfaat}
Adapun manfaat dari penelitian ini adalah sebagai berikut:
\begin{enumerate}[noitemsep]
    \item Mengetahui cara melakukan oversampling dengan metode SMOTE dan ADASYN
    \item Mengetahui cara membuat model Random Forest dan Advanced Gradient Boosting Techniques seperti XGBoost.
    \item Mengetahui metode oversampling terbaik diantara SMOTE dan ADASYN didalam sebuah algoritma Random Forest dan XGBoost untuk membuat model pendeteksi penipuan kartu kredit.
\end{enumerate}

\section{Sistematika Penulisan} \label{I.Sistematika}
Adapun sistematika penulisan dalam penelitian adalah sebagai berikut:
\subsection{Bab I}
Bab ini merangkum beberapa hal yang berkaitan dengan penelitian yakni latar belakang, pengenalan masalah, perumusan pertanyaan penelitian, pendekatan metodologi, sasaran tujuan penelitian, batasan–batasan cakupan penelitian, dampak manfaat hasil penelitian, dan susunan sistematis penulisan.
\subsection{Bab II}
Bab ini memfokuskan pada pengulasan tinjauan pustaka yang terkait dengan topik yang meliputi kartu kredit, SMOTE, ADASYN, Random Forest, XGBoost dan serta metode pengujian.
\subsection{Bab III}
Bab ini menguraikan tentang metodologi yang akan digunakan dalam penelitian. Ini meliputi pendekatan pembuatan model seperti pembuatan program, pengumpulan dataset, persiapan data, implementasi dan evaluasi terhadap program yang telah dirancang.
\subsection{Bab IV}
Bab ini menjelaskan hasil dari penelitian yang telah dilakukan dan berisikan pembahasan dan analisis secara detail tentang kinerja metode oversampling SMOTE dan ADASYN dalam meningkatkan akurasi model pendeteksi penipuan kartu kredit.
\subsection{Bab V}
Bab ini berisikan pembahasan tentang kesimpulan berdasarkan hasil dan pembahasan dari penelitian yang telah dilakukan dan juga terdapat saran yang dipaparrkan untuk pengembangan penelitian di masa yang akan datang. 
